\documentclass[12pt, a4paper, twoside, titlepage]{article}

\usepackage{answers}
\usepackage{setspace}
\usepackage{graphicx}
\usepackage{enumitem}
\usepackage{multicol}
\usepackage{mathrsfs}
\usepackage{german}
\usepackage[margin=1in]{geometry} 
\usepackage{amsmath,amsthm,amssymb}
 
\newcommand{\N}{\mathbb{N}}
\newcommand{\Z}{\mathbb{Z}}
\newcommand{\C}{\mathbb{C}}
\newcommand{\R}{\mathbb{R}}

\DeclareMathOperator{\sech}{sech}
\DeclareMathOperator{\csch}{csch}
 
\newenvironment{theorem}[2][Theorem]{\begin{trivlist}
\item[\hskip \labelsep {\bfseries #1}\hskip \labelsep {\bfseries #2.}]}{\end{trivlist}}
\newenvironment{definition}[2][Definition]{\begin{trivlist}
\item[\hskip \labelsep {\bfseries #1}\hskip \labelsep {\bfseries #2.}]}{\end{trivlist}}
\newenvironment{proposition}[2][Proposition]{\begin{trivlist}
\item[\hskip \labelsep {\bfseries #1}\hskip \labelsep {\bfseries #2.}]}{\end{trivlist}}
\newenvironment{lemma}[2][Lemma]{\begin{trivlist}
\item[\hskip \labelsep {\bfseries #1}\hskip \labelsep {\bfseries #2.}]}{\end{trivlist}}
\newenvironment{exercise}[2][Exercise]{\begin{trivlist}
\item[\hskip \labelsep {\bfseries #1}\hskip \labelsep {\bfseries #2.}]}{\end{trivlist}}
\newenvironment{solution}[2][Solution]{\begin{trivlist}
\item[\hskip \labelsep {\bfseries #1}]}{\end{trivlist}}
\newenvironment{problem}[2][Problem]{\begin{trivlist}
\item[\hskip \labelsep {\bfseries #1}\hskip \labelsep {\bfseries #2.}]}{\end{trivlist}}
\newenvironment{question}[2][Question]{\begin{trivlist}
\item[\hskip \labelsep {\bfseries #1}\hskip \labelsep {\bfseries #2.}]}{\end{trivlist}}
\newenvironment{corollary}[2][Corollary]{\begin{trivlist}
\item[\hskip \labelsep {\bfseries #1}\hskip \labelsep {\bfseries #2.}]}{\end{trivlist}}

\title{Modellierung von Kraft-W\" arme-Kopplungs-Anlagen \\
Anbindung geologischer W\" armespeicher}
\begin{document}

\maketitle
 

\section{Geologische Speicher}
%
Modelliert wird das r\" aumliche und zeitliche Verhalten von Str\" omung und W\" arme einschlie\ss lich deren Wechselwirkungen.
%
Das Modellgebiet umfasst den Speicher und das Speicherumfeld.
%
Das Druck- und Temperaturfeld werden aus einem Initalzustand heraus berechnet (z.B. aus dem urspr\" unglichen und ungest\" orten Zustand oder aus dem Ende einer Belandungsphase heraus).
%
Die Entwicklung beider Felder wird \" uber Austauschraten und Randbedingungen gesteuert
(z.B. der Speicherleistung, Landoberfl" achentemperatur, Abfl\" usse).
%
Die Parameter sind stets messbare Gr\" ossen.
%
%
\subsection{Bilanzgleichungen} 
%
%
\noindent
Druckfeld $p = p(x, y, z, t)$ und Temperaturfeld $T = T(x,y,z, t)$ wird mit den folgenden Bilanzgleichungen f\" ur Masse (\ref{eqn:bilanz_masse}) und innerer Energie (\ref{eqn:bilanz_innereEnergie}) berechnet
%
\begin{subequations}
\begin{align}
S_s\frac{\partial p}{\partial t} + \nabla \cdot \vec q^L &= \tilde q^L
\label{eqn:bilanz_masse} \\
C\frac{\partial T}{\partial t} + \nabla \cdot \left(\vec q_{adv}^H +  \vec q^H \right ) &= \tilde q^H
\label{eqn:bilanz_innereEnergie}
\end{align}
\end{subequations}
%
Externe Quellen und Senken werden \" uber die Terme $\tilde q^L$ und $\tilde q^H$ ber\" ucksichtig.
\noindent
Der Speicherkoeffizient $S_s$ und die W\" armekapazit\" at $C$ sind wie folgt
%
\begin{subequations}
\begin{align}
S_s &= \phi \alpha_l + (1-\phi) \alpha_s \\
C &= \phi \rho_l c_l + (1-\phi) \rho_s c_s
\end{align}
\end{subequations}
%
Der Index $l$ bezeichnet einen Fl\" ussigkeitsparameter und der Index $s$ einen Festphasenparameter.
%
Der Parameter $\phi$ gibt den Volumenanteil der Nicht-Festphase an
(Porosit\" at im Aquifer, $\phi=1$ innerhalb einer Sondenverrohrung).
%
Der Parameter $\alpha$ kenzeichnet die Kompressibilit\" at der entsprechenden Phase,
%
$\rho$ die Dichte und $c$ die spezifische Kapazit\" at.
%

\noindent
Geologische Speicher bestehen im Wesentlichen aus einer W\" armetr\" agerfl\" ussigkeit und aus Material (Sedimente, Verrohrung, Isolation, etc.).
%
Weitere Fl\" ussigkeiten k\" onnen mitsimuliert werden, um dernen Einfl\" usse zu ber\" ucksichtigen (z.B. W\" armeverluste durch Grundwasser).
%
Weiterhin k\" onnen Gase eine Rolle spielen (z.B. die teilweise wasserges\" attigte Bodenzone).
%
Gase werden bei der Speichersimulation als immobil angenommen und die Festphasenparameter entsprechend angepasst.
%
Allerdings unterst\"utzt die zugrundeliegende Implementation auch die Simulation von Gasbewegungen und Phasen\" uberg\" angen.
%
\subsubsection{Diffusion und Dispersion}
%
\noindent
Die folgenden Gradientenfl\" usse treten auf
%
\begin{subequations}
\begin{align}
\vec q^L &= - \frac{\bold k}{\mu} \nabla\left( p + \rho_l g z \right)
\label{eqn:flussStroemung} \\
\vec q^H &= -\left( \bold \Lambda + \rho_l c_l \bold D \right) \nabla T
\label{eqn:flussWaerme}
\end{align}
\end{subequations}
%
wobei $\mu$ die dynamische Viskosit\" at und $g$ die Gravitationskonstante sind.
%
Der Permeabilit\" atstensor ist in diesem Zusammenhang eine Diagonalmatrix
$k_{ij} = k_i \delta_{ij} $ ($i$, $j$ = $1$, $2$, $3$)
und die W\" armeleitf\" ahigkeit isotrop $\Lambda_{ii} = \lambda$, wobei
$\lambda = \phi \lambda_l + (1-\phi) \lambda_s$.
%
Bei Bedarf k\" onnen im Speichermodell auch vollbesetzte Leitf\" ahigkeistensoren verwendet werden.

\noindent
Der Durchfluss (\ref{eqn:flussStroemung}) wird in der Hydrogeologie Darcy-Gesetz genannt und der W\" armefluss
(\ref{eqn:flussWaerme}) beinhaltet das Fourier-Gesetz. 
%
Wird ein por\" oses Medium (z.B. ein Speichersediment) durchstr\" omt, so tritt zus\" atzlich Dispersion auf
%
\begin{eqnarray}
D_{ij} = \alpha_l | \vec q^L | \delta_{ij} + \left( \alpha_t - \alpha_l \right) \frac{q_i^L q_j^L}{| \vec q^L | }  
\label{eqn:dispersion}
\end{eqnarray}
%
wobei $\alpha_l$ die Dispersionsl\" ange in Str\" omungsrichtung (longitudinal) und 
$\alpha_t$ die Dispersionsl\" ange senkrecht zur Str\" omungsrichtung (transversal) sind.
%
\subsubsection{Konvektion}
%
\noindent
Str\" omung transportiert W\" arme gem\" a\ss
%
\begin{eqnarray}
\vec q_{adv}^H = \rho_l c_l \vec q^L T
\label{eqn:advektion}
\end{eqnarray}
%
\noindent
W\" arme beeinflu\ss t Str\" omung \" uber die Dichte $\rho_l = \rho_l(T)$ und Viskosit\" at $\mu = \mu(T)$.
%
Gleichung (\ref{eqn:flussStroemung}) berechnet den Auftrieb und es kann zu freien Konvektionsstr\" omungen kommen.
%

%

\subsubsection{L\" osungsans\" atze}
%

%
\begin{figure} [b!]
\centering
\includegraphics[width=50mm]{nnnc.png}
\caption{Kopplung von BTES-Sondenstr\" angen \" uber einen thermischen NNNC-Flu\ss \ $Q_T$. Vollst\" andiger Rundlauf (links) und NNNC-gekoppelte Str\" ange (rechts).}
\label{fig:nnnc}
\end{figure}
%

%
\noindent
Es erfolgt eine Finite-Element-Diskretisierung des Modellgebietes. 
%
Das Bilanzgleichungssystem ist ein Anfangswert-Randwert-Problem und wird zeitschrittweise gel\" ost.
%
Die enthaltenden Gleichungen f\" ur Str\" omung (\ref{eqn:bilanz_masse}) und W\" arme (\ref{eqn:bilanz_innereEnergie}) werden sequentiell gel\" ost und bei Bedarf \" uber diese iteriert.

%
Diskretisierungsknoten aus unterschiedlichen Regionen im Modellgebiet k\" onnen durch Austauschfl\" usse (Non-Neighbor Node Connections - NNNCs) verkn\" upft werden.
%
Somit lassen sich beispielsweise Sondenstr\" ange in BTES-Simulationen verbinden
und der Vernetzungs- und Rechenaufwand deutlich reduzieren (Abbildung \ref{fig:nnnc}).
%
%


\subsection{Gesamtspeichersimulationsans\" atze}
%
\begin{figure} [b!]
\centering
\includegraphics[width=150mm]{speicher.png}
\caption{(a) ATES-Brunnendoublette; (b) BTES-Sondenstrangpaar. Ein BTES besteht aus einer Vielzahl an W\" armesonden (Bohrungen), die jeweils Sondenstr\" ange enthalten.}
\label{fig:speicher}
\end{figure}
%


Die Speicherung und der Entzug von W\" arme erfolgt bei Aquiferw\"armespeichern (ATES-Systemen) mittels Injektions- und Extraktionsbrunnen (Brunnendoubletten).
%
\" Uber die Brunnen zirkuliert ein Speichermedium (Wasser) durch ein Aquifer und eine Installation an der Landoberfl\" ache.
%
In Erdw\" armesondenfeldern (BTES-Systemen) befindet sich eine Vielzahl an
Erdw\" armesonden von einigen zehner Metern L\" ange innerhalb von vertikalen Bohrungen.
%
Eine W\" armetr\" agerfl\" ussigkeit zirkuliert durch die Verrohrung.
%
Aquiferspeicherbeladung und Entladung erfolgt durch Zirkulation der W\" armetr\" agerfl\" ussigkeit in entgegengesetzer Str\" omungsrichtung.
%
Bei Erdw\" armefeldern wird \" ublicherweise ebenfalls die Str\" omungsrichung zur Entladung umgekehrt, so dass das W\" armetr\" agerfluid im Bereich des Speicherzentrums w\" armer als am Speicherrandbereich ist.
%
Damit lassen sich laterale W\" armeverluste verringern.

%
\subsubsection{Aquiferw\" armespeicher - ATES-Systeme}
%
%
Die Durchfl\" usse innerhalb der Brunnenbohrungen und innerhalb sonstiger Verrohrungen werden im Speichermodell durch L\" osen der Bilanzgleichungen (\ref{eqn:bilanz_masse}) und (\ref{eqn:bilanz_innereEnergie}) in 1D berechnet.
%
Der Einfluss von Rohrkr\" ummungen auf das Str\" omungsverhalten wird nicht ber\" ucksichtigt.
%
Zur Vereinfachung k\" onnen Rohrstr\" ange mit NNNC verkn\" upft werden (Abb. \ref{fig:speicher}(a)). 
%
Es wird das Druck- und Temperaturfeld im Speicheraquifer und zumindest in den angrenzenden Stauschichten in 3D berechnet.
%


%
Im Speicheraquifer tritt gew\" ohnlicherweise eine Temperaturschichtung auf (Warmes Wasser \" uber dem kalten Wasser).
%
Daher sind Kopplungsiterationen zwischen den beiden Bilanzgleichungen erforderlich.
%
Besteht eine nat\" urliche Grundwasserstr\" omung (Hintergrundstr\" omung), so beeinflu\ss t sie die R\" uckgewinnungsrate im Speicherbetrieb.
%
Sie wird mittels Str\" omungsquellen- und Senkenterme $\tilde q^L$, Druckrandbedingungen und (oberstromige) Temperaturrandbedingungen im Modell erzeugt.
%
 
%
\subsubsection{Erdw\" armesondenfelder - BTES-Systeme}
%
%
Das Temperaturfeld wird in 3D abgebildet.
%
Somit werden die thermischen Wechselwirkungen zwischen den Sondenstr\" angen innerhalb einer Erdw\" armesonde detailiert abgebildet. 
%
Sie spielen eine wichtige Rolle, wohingegen das Str\" omungsprofil vereinfacht in 1D abgebildet wird ($\phi = 1$, $S_s = 0$).
%
Sondenstr\" ange innerhalb von einer Bohrung oder \" uber Bohrungen hinweg werden \" uber NNNC gekoppelt (Abb. \ref{fig:nnnc}, \ref{fig:speicher}(b)).

%
Speichermedium sind die Sedimente einschlie\ss lich der darin enthaltenen Fl\" ussigkeiten (Grundwasser).
%
Der W\" armeaustausch zwischen dem W\" armetr\" agerfluid in der Verrohrung und dem Speichermedium erfolgt konduktiv.
%
Steile Temperaturgradienten im Nahbereich der Sonden erfordern eine hohe zeitliche und r\" aumliche Aufl\" osung in den Simulationen.
%

%
Eine Grundwasserstr\" omung an oder im Umfeld von Erdw\" armesonden kann die R\" uckgewinnungsrate beeinflussen und wird gegebenfalls \" uber Quellen- und Senkenterme bzw. Randbedingungen ber\" ucksichtigt.
%
Eine wichtige Rolle spielen W\" armeverluste \" uber die Isolationsschicht an der Landoberfl\" ache.
%
Diese werden im Modell \" uber eine Temperaturrandbedingung erzeugt.
%
W\" armeverluste in laterale oder tiefere Regionen werden ebenso durch Temperaturrandbedingungen im Modell erzeugt. 
 
\subsection{Steuerung des Speicherbetriebes}
%
Der Speicherbetrieb wird im Modell \" uber Randbedingungen und Austauschraten (Quell- und Senkenterme) gesteuert.
%
Unabh\" angig von der Art des W\" armespeichers (ATES oder BTES) wird zur L\" osung der W\" armebilanzgleichung (\ref{eqn:bilanz_innereEnergie}) die Temperatur der einstr\" omenden Fl\" ussigkeit $T_{in}$ ben\" otigt (Abb. \ref{fig:speicher}).
%
Es mu\ss \ ferner entweder die Flie\ss rate $q^L$ vorgegeben sein oder es mu\ss \ sich diese aus alternativen Vorgaben invers bestimmen lassen.
%
%
Die Speicherleistung (Einspeicherrate $\hat q^T >0$ beziehungsweise die Entnahmerate $\hat q^T < 0$) ergibt sich aus der Temperaturdifferenz zwischen den einstr\" omenden und ausstr\" omenden Fl\" ussigkeiten
%
\begin{eqnarray}
\hat q^{T} = q^L \left[ (c_{l} \rho_{l} T)^{SP}_{in} - (c_{l} \rho_{l} T)^{SP}_{out}\right]
\label{eqn:leistung}
\end{eqnarray}
%
Ist eine gewisse Speicherleistung $\hat q^{T}$ vorgegeben, so wird daraus die Temperatur der ausstr\" omenden Fl\" ussigkeit $T^{SP}_{out}$ und eine Flie\ss rate 
%
\begin{eqnarray}
\hat q^L = \frac{\hat q^{T}}{ (c_{l} \rho_{l} T)^{SP}_{in} - (c_{l} \rho_{l} T)^{SP}_{out}}
\end{eqnarray}
%
berechnet.
%
Die Berechnung erfolgt \" uber eine Iteration \" uber die Bilanzgleichungen f\" ur die Str\" omung (\ref{eqn:bilanz_masse}) und W\" arme (\ref{eqn:bilanz_innereEnergie}). 


Zus\" atzlich bestehen Zwangsbedingungen in Form von Schwellen
%
\begin{subequations}
\begin{align}
q^L &\leq q_{max}^L
\label{eqn:schwelleFliessrate} \\
 p_{min} &\leq p \leq p_{max}
\label{eqn:schwelleDruckfeld} \\
 T_{min} &\leq T \leq T_{max}
\label{eqn:schwelleTemperaturfeld}
\end{align}
\end{subequations}
%
So ist die Flie\ss rate beispielsweise durch eine maximale Pumprate und das Temperaturfeld durch einen Gefrierpunkt beschr\" ankt.
%
Wird eine dieser Schwellen w\" ahrend des Speicherbetriebes \" uberschritten, so l\" asst sich im Modell die Fliessrate entsprechend anpassen und die Simulation mit der maximal m\" oglichen Speicherleistung fortsetzen.

\subsection{Proxy-Ans\"atze}

Gesamtspeichersimulationen bedingen hohe Modellerstellungs- und Rechenresourcen. Daher werden Proxy-Modelle verwendet.

\begin{figure} [b!]
\centering
\includegraphics[width=150mm]{proxy_ATES.png}
\caption{Proxy-Modellansatz zur effizienten Absch\" atzung der Leistung (Raten, Temperaturen) eines Aquiferw\" armespeichers.}
\label{fig:proxy_ATES}
\end{figure}


\begin{figure} [b!]
\centering
\includegraphics[width=150mm]{proxy_BTES.png}
\caption{Proxy-Modellansatz zur effizienten Absch\" atzung der Speicherleistung (Raten, Temperaturen) eines Erdw\" armesondenfeldes.}
\label{fig:proxy_BTES}
\end{figure}



\subsubsection{Aquiferw\" armespeicher - ATES-Systeme}
%
Bei hinreichend gro\ss em Brunnenabstand erfolgt die W\" armeausbreitung im Nahfeld der Brunnen zylindrisch.
%
In diesem Fall kann die Einspeicher- und Entzugsleistung (Gleichung \ref{eqn:leistung})
mit axialsymmetrischen 2D-Modellen berechnet werden (Abbildung \ref{fig:proxy_ATES}). 
%
Dichtevariationen und W\" armeverluste in angrenzende Stauschichten werden dabei ber\" ucksichtigt.
%
Der Einfluss einer nat\" urlichen Grundwasserstr\" omung (Hintergrundstr\" omung) bleibt dabei unber\" ucksichtigt.

\subsubsection{Erdw\" armesondenfelder - BTES-Systeme}
%
Es wird lediglich eine Einzelsonde im Speicherbetrieb simuliert und angenommen, dass sich die Sonden in einem Gesamtspeicher gleichartig verhalten und gegenseitig nicht beeinflussen (Spiegelsondenansatz).
%
Aus der errechneten Leistung der Einzelsonde (Gleichung \ref{eqn:leistung}) ergibt sich der Sondenbedarf des Gesamtspeichers.
%
Dar\" uber hinaus lassen sich aus dem errechneten W\" armefeld die lateralen W\" armeverluste absch\" atzen.

%
\section{Kraftwerkmodell}
%
\subsection{W\" armetauscher}
%

%
\begin{subequations}
\begin{align}
q^L \left[ (c_l \rho_l T)^{WP}_{in,1} - (c_l \rho_l T)^{WP}_{out,1} \right] &= q^L \left[ (c_l \rho_l T)^{WP}_{out,2} - (c_l \rho_l T)^{WP}_{in,2} \right] \\
Q^T_{tausch}  &=  k A_q \Delta T_m
\end{align}
\end{subequations}
%
Der W\" armeaustausch $Q^T_{tausch}$ ergibt sich aus der effektiven mittlere Temperaturdifferenz
$\Delta T_m = \Delta T_m (T^{WP}_{in,1}, T^{WP}_{out,1}, T^{WP}_{in,2}, T^{WP}_{out,2})$,
die von der Stromf\" uhrung im W\" armetauscher (z.B. Gleich-, Gegen-, Kreuzstrom) abh\" angt.
%
Austauschparameter sind der W\" armedurchgangskoeffizient $k$ und die W\" arme\" ubertragungsfl\" ache $A_q$.

\section{Modellschnittstelle}
%

\begin{figure} [b!]
\centering
\includegraphics[width=100mm]{speicheranbindung.png}
\caption{Speicheranbindung.}
\label{fig:speicheranbindung}
\end{figure}


\begin{figure} [b!]
\centering
\includegraphics[width=150mm]{waermenetz.png}
\caption{W\" armenetz.}
\label{fig:waermenetz}
\end{figure}


\end{document}

